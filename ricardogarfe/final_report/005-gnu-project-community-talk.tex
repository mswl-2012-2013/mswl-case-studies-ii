\section{GNU}
\label{sec:gnu}

Starting with the acronym inception definition of GNU \textit{"GNU is not Unix!"} in 1983. GNU is divided in three departments:
\begin{itemize}
	\item \textit{The GNU Project}.
	\item \textit{GNU Software} - Develops GNU software.
	\item \textit{The GNU System} - GNU Software + external Software.
\end{itemize}

\par GNU selected Unix because of simplicity instead of VMS. Composed by different independent utilities. Decoupled and modular, so the choice to clone the system.

\par \textit{GNU Project} is a Global Project, hasn’t legal entity. RMS (Richard Matthew Stallman) created \textit{Free Software Foundation} (FSF) to convert legal resources to GNU hackers.

\subsection{Techonologies}

\par All projects inside GNU universe share a homogeneous structure and tools to contribute. A part from common techonoligies they use specific tools for each project, depending of the tool purpose.

\par Here is a common list for every GNU project:

\begin{itemize}
	\item Webpage
	\item Development
	\item Manual
	\item Reporting bugs
	\item Getting help
\end{itemize}

\par We apply this structure to an existing GNU project: \href{http://www.gnu.org/software/recutils/}{GNU Recutils}and see how easy is to get all starter information.

\textit{GNU Recutils} is a set of tools and libraries to access human-editable, plain text databases called recfiles.

\begin{itemize}
	\item Webpage -\url{http://www.gnu.org/software/recutils/}{http://www.gnu.org/software/recutils/}
	\item Development -\url{http://savannah.gnu.org/projects/recutils/}{http://savannah.gnu.org/projects/recutils/}
	\item Manual -\url{http://www.gnu.org/software/recutils/manual/}{http://www.gnu.org/software/recutils/manual/}
	\item Reporting bugs -\url{http://savannah.gnu.org/bugs/?group=recutils}{http://savannah.gnu.org/bugs/?group=recutils}
	\item Getting help - FAQ\url{http://www.gnu.org/software/recutils/faq.html}{http://www.gnu.org/software/recutils/faq.html}
	\item Mailing lists - bugs\url{http://lists.gnu.org/mailman/listinfo/bug-recutils} and help\url{http://lists.gnu.org/mailman/listinfo/help-recutils}.
\end{itemize}

\par You have total freedom from the moment when develop, test, and evolve Recutils from tools or guides that offer.

\par At last but not least important: Following \href{http://www.gnu.org/prep/standards/}{GNU coding standards} that are very important to develop readable code.

\subsection{How to Contribute}

\par At this section you will wonder, how I can contribute in a GNU project? one of the most important projects in history?It's easier than it at first seems, of course, following some basic rules of coordination and participation.

\par In an overall picture youcan recognize this structure:

\begin{itemize}
	\item \textit{RMS} - Appoints maintainers. \textit{"GNUism"}.
	\item \textit{GNU Advisory Committe} (2009) - Advisory, communication, transversal vision, external contact. advisory@gnu.org. 8 people, most maintainers and FSF America and Europe vice-presidents.
	\item \textit{Maintainers} - Maintainer, lead program development. Follow \href{http://www.gnu.org/prep/maintain_toc.html}{GNU policies for maintainers}.
	\item \textit{Contributors} - Developers with or without commit access.
\end{itemize} You start in the last point as a contributor with ot without commit access, because GNU is not only writting code. There are many areas: documentation, translations, blogging, law, in which where you can be more helpfull that you think. Take a tour in \href{http://www.gnu.org/help/}{How to help GNU}.

\par But, focussing strictly on development you start as a non commit contributor, inside a mailing list following basic guides. Explaining Recutils project guidelines:

\begin{itemize}
	\item \textit{Test releases} - Trying the latest test release (when available) is always appreciated. Test releases of Recutils can be found at http://alpha.gnu.org/gnu/recutils/ (via HTTP) and ftp://alpha.gnu.org/gnu/recutils/ (via FTP).
	\item \textit{Development} - For development sources, bug and patch trackers, and other information, please see the Recutils project page at savannah.gnu.org.
	\item \textit{Translating Recutils} - To translate Recutils's messages into other languages, please see the Translation Project page for Recutils. If you have a new translation of the message strings, or updates to the existing strings, please have the changes made in this repository. Only translations from this site will be incorporated into Recutils. For more information, see the Translation Project.
	\item \textit{Maintainer} - Recutils is currently being maintained by Jose E. Marchesi. Please use the mailing lists for contact.
\end{itemize}

\par Starting from these test releases to get an inmersion in the project and know its architecture. This starting point is very important for FLOSS project because is an starting mentoring guide through the code and project structure.

\par After this step you can know how to read (better not perfectly) the code, understand bugs, follow solutions and develop your first patch. Here is your gateway to contribute with code to a GNU Project and the path to become a maintainer, only if you want to learn (for 'gratis').

\par Other ways to contribute, as I explained before, are translate projects and develop documentation. Jos\'e E. Marchesithat explains what is and what gives you 'Software Libre': \href{http://es.gnu.org/~jemarch/bicicletas.html}{Bicicletas y Software Libre}.

\begin{quotation}
    \emph{Ayer me compr\'e una bicicleta. ¡Y es algo fant\'astico! Puedo utilizarla tanto en la ciudad como en el campo. ¡Y en cualquier mes del año! Puedo darme paseos y, si alg\'un d\'ia me hace falta, puedo repartir peri\'odicos y sacarme unos durillos. Mi bici no tra\'ia luces, pero no es problema: le he puesto una dinamo y una buena luz, incluso lo he hecho yo mismo: ¡ahorras y aprendes al mismo tiempo! Adem\'as la hemos pagado a medias entre mi compañera y yo. No es ning\'un problema compartirla. Pero quiz\'a alg\'un d\'ia nos cansemos y queramos librarnos de ella. Podremos venderla de segunda mano. O se la regalaremos a alguien que la vaya a usar.
    \\ Hoy me he comprado un programa de ordenador. No es nada muy sofisticado (un cliente para leer el correo), pero tiene buena pinta y hace mogoll\'on de cosas. Me ha salido bastante caro porque no he podido comprarlo a medias. Me dicen que ella no puede utilizarlo, aunque lo necesita. Adem\'as s\'olo podr\'e utilizarlo durante un año. Me han dicho que pasado ese tiempo debo pagarlo otra vez y usarlo durante un año m\'as y no puedo engañarles: el programa se rompe y ya no funciona m\'as. Curiosamente he le\'ido que si utilizo este programa en Corea del Norte ser\'a ilegal. ¡Menos mal que no planeo viajar por all\'i! Esta compra no me va a salir nada rentable porque tampoco debo utilizarlo para ganar dinero. “Uso no comercial”, dicen.
    \\ Luego en casa me he dado cuenta de que el programa usa una letra muy pequeña y no veo bien las cosas. Se lo he llevado a un primo m\'io inform\'atico a ver si me lo pod\'ia arreglar, pero me ha dicho que no puede, que es ilegal. Ya no quiero el programa (estoy muy enfadado) pero no puedo venderlo ni regal\'arselo a nadie: tambi\'en est\'a prohibido.
    \\ Me dice mi primo que esto se llama “modelo de mercado del software privativo” y que lleg\'o a principios de los ‘80. Pero dice que existe una alternativa llamada software libre. “¿Y c\'omo es eso?”, le pregunt\'e. “F\'acil”, contest\'o, “es como comprarte una bicicleta”.}
\end{quotation}

% section gnu (end)