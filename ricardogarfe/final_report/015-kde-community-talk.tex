\section{KDE}
\label{sec:kde}

\par KDE is a Desktop ? an environment ? What is KDE ? KDE started being a Desktop environment. Was founded in 1996 by Matthias Ettrich looking for the creation of a working and intuitive Desktop Environment for Unix distributions with common applications to create an ecosystem.

\par The projects evolved using Qt framework which became FLOSS software in 1998 and the first version KDE 1.0.

\par Continued evolving and growing with user until this transformation in 2009 into a FLOSS developers community, not only Desktop environment. Thus KDE wanted to spread the use of Free Software in desktops, without focusing on the development of the desktop environment. The foundation covers most fields of product development for desktop. Thus trying to spread the use of Free Software in all levels of users. It is an ambitious but successful.

\subsection{Technologies}

\par Development in KDE is divided in different sections:

\begin{itemize}
	\item \textit{Core Tools} - CMake and VCS tool, Git or SVN. Basic tools to build.
	\item \textit{Debugging and Analysis} - Tools for debugging KDE projects are composed by Valgrind, The GNU Project Debugger (GDB), KDbg and DDD, MS Windows tools.
	\item \textit{Development Tools} - A set of IDEs to develop KDE projects: Qt Creator, KDevelop, MS Visual Studio® Express IDE (Windows only)
	\item \textit{Internationalization (i18n) Tools} - Set of internationalization tools to contribute easily - Lokalize, Dr. Klash, The x-test language.
\begin{itemize}
	\item \textit{Examining .po files}
\end{itemize}
	\item \textit{Helper Tools} - Get information about KDE's installation - kde-config, Driving Konqueror From Scripts - kfmclient, Updating User Configuration Files - kconf\_update, Generating apidox, Automoc4, svnmerge.py
	\item \textit{Quality Assurance} - Code Review, Continuous Building, English Breakfast Network - Static Analysis
\end{itemize} Detailed information can be found in \href{http://techbase.kde.org/Development/Tools}{http://techbase.kde.org/Development/Tools}. TechBase KDE is a place made to share knowledgement for everyone. Was created after KDE became more than a Desktop Client (first edition of Development Tools in 2006).

\subsection{How to contribute}

\par How to become a committer ? You can ask any community member for this permission. It's a community with easy entry and likes to receive new members.

\par After this point, you have to work on it following community guidelines. The steps aren't different from other communities, we can see a common pattern in communities:

\begin{itemize}
	\item \textit{News and Mail Sources} - Mailing lists, history and news.
	\item \textit{Reporting Bugs} - Bug reporting and bug fixing.
	\item \textit{Getting Started with Coding}:  C++, Qt,  KDE - Yes, you have to code to contribute to development.
	\item \textit{Getting Involved in Bug Hunting and Application Quality} - Be a Quality Assurance expert helping improving test and quality. 
	\item \href{http://community.kde.org/KDE/Junior_Jobs}{\textit{Junior Bugs}} - This is a special part to get in touch with the community, an easy way to start fixing bugs and know how to work in community.
	\item \textit{User Interface} - Avoid flames about UI \textit{"User interface is a very wide subject".}
	\item \textit{Getting Answers to Your Questions} - Learn to search and use documentation. It's very important to read the FAQs, mail lists, documentation, wikis, bugs before post anything because could be duplicated and become a nonsense effort
\end{itemize}

\par Extended guide could be found at \href{http://techbase.kde.org/Contribute}{http://techbase.kde.org/Contribute}.

% section kde (end)