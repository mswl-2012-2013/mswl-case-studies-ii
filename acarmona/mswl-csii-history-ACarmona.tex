%%WL-INT  29-09-2011
\documentclass[a4paper,oneside,11pt]{article}
\usepackage[spanish]{babel}
\usepackage[utf8]{inputenx}
\usepackage[left=3cm,top=3cm,right=3cm,bottom=3cm,nohead,nofoot]{geometry}
\usepackage{graphicx}
\usepackage{url}
\usepackage{placeins}
\begin{document}
\vspace{5cm}

\begin{titlepage}
\begin{center}
% Upper part of the page
%\textsc{\LARGE Universidad Rey Juan Carlos}\\[1.5cm]
\includegraphics[width=50mm,height=20mm]{./latex/logo-urjc.jpg}

%\vspace{0.5cm}
\vspace{4cm}
\textsc{\Large Master Software Libre 2012/2013}\\[0.5cm]
\textsc{\Large MSWL-CSII}\\[0.5cm]
% Title --
%\HRule \\[0.4cm]
\hrule height 2pt
\begin{center}
{ \huge \bfseries Historia Proyectos}
\end{center}
\hrule height 2pt
\vspace{0.4cm}
%\HRule \\[1.5cm]
% Author and supervisor
\begin{minipage}{0.4\textwidth}
\large
\begin{center}
Antonio Carmona \'Alvarez Mar-2013
\end{center} 
\end{minipage}
\vfill
% Bottom of the page
\includegraphics[width=3.1cm,height=1.08cm]{./latex/by_petit.png}
{\large}
\end{center}
\end{titlepage}

\newpage
\tableofcontents  %%Para que funcione el indice hay que ejecutarlo dos veces elpdflatex
\newpage

\section{Apache}

El desarrollo del servidor web Apache comenz\'o en el año 1995, inicialmente se bas\'o en en el c\'odigo fuente del servidor web NCSA
pero fue reescrito por completo de nuevo y estructurado en m\'odulos. Para proteger al software desarrollado y dotarlo de una
infraestructura legal que lo soportara y le diera cobertura se cre\'o en 1999 la ASF, la Apache Software Foundation, es una
organizaci\'on sin \'animo de lucro formada por voluntarios.\\
\\
Todo surgi\'o cuando el grupo de desarrolladores que ven\'ia trabajando en el servidor web NCSA decidi\'o dejar el proyecto para desarrollar Netscape.
Dado que hab\'ia bastantes usuarios que utilizaban dicho servidor web, desde los foros y listas de correo del proyecto NCSA se fueron organizando con 
el fin de retomar el proyecto haciendo un fork del proyecto original NCSA, empezaron simplemente por ir parcheando el proyecto, pero
fue creciendo poco a poco, entre 1995 y 1999 creci\'o de manera significativa y hab\'ia ya una considerable cantidad de versiones liberadas. 

\section{Webkit}

Webkit, es un motor de rendering de web, no es un navegador web, surge como un fork de KHTML (proyecto de renderizado HTML de KDE )
 y de KJS (motor de Javascript de KDE), fue hecho por Apple en 2001, ambos proyectos (KHTML y KJS) fueron portados a MACOS X  y renombrados como Webcore y 
JavaScriptCore respectivamente, ambos eran software libre (GPL) y eran parte del proyecto WebKit el cu\'al inicialmente era privativo pero fue 
liberado en 2005 abriendo su c\'odigo fuente y haciendo p\'ublica su lista de bugs, Webkit se liber\'o con licencia BSD.\\
\\
En el 2010 se anunci\'o la salida de la versi\'on Webkit2 , que fue reescrito desde cero para conseguir un procesamiento dividido
del renderizado de cada uno de los elementos de la p\'agina web (html, javascript..). Fue en este año cuando se celebr\'o la primera 
conferencia Webkit.

\section{Plan 9}

A mediados de los 80 el mismo grupo de desarrolladores (en la empresa Bell labs) que realiz\'o Unix empezaron a desarrollar un nuevo sistema operativo llamado
Plan9 que le reemplazara, lo iniciaron en la misma sala donde se desarroll\'o Unix, la raz\'on por la que se iniciaron en el desarrollo
de un nuevo sistema operativo era porque Unix no estaba pensado para distribuirse en red ( esto se consigui\'o posteriormente con la pila tcp/ip)
su kernel no era un diseño para estar integrado en una red, estaba pensado para una m\'aquina
multiusuario muy grande, en los 80 llegaron las redes y empezaron los problemas , interfaces duplicadas , el kernel empez\'o a crecer exageradamente y todo porque el 
diseño original no era bueno.
\\
Plan9 se inici\'o desde cero sin compatibilidad hacia atr\'as
\\
Sus caracter\'isticas principales son:

\begin{itemize}  
	\item Todo es un fichero, incluso la red /dev/net
	\item Todos los ficheros se exportan a trav\'es de la red a traves de un protocolo llamado 9P, que es parecido
	 a NFS y vale para ficheros de datos o ficheros que no son de datos de cualquier otro tipo. 
	\item El kernel es un multiplexor de servidores  de ficheros que hablan  9P, todo habla 9P ya sea dentro del
	 servidor o con otros fuera.
	\item Los procesos tienen su propio espacio de nombres
	\item Las m\'aquinas están especializadas, servidores de cpu, autenticaci\'on, ficheros, terminales de usuario)
\end{itemize}

La primer versi\'on de Plan9 fue en 1992 para universidades, se tuvo mucho cuidado en que fuera c\'odigo sencillo
y minimalista y que fuera muy f\'acil de portar a cualquier hardware.
En el año 2000 se pudo descargar de manera gratuita con licencia Plan9 , no era licencia considerada
como FLOSS, Stallman dijo que la licencia plan9 no era libre entre otras cosas porque obligaba
a redistribuir bajo contrato, en el 2002 se relicenci\'o con la licencia Lucent Public License v1.02 que ya era 
una licencia libre reconocida por la OSI y la FSF, cumple adem\'as las guidelines de debian,
 es estilo MIT y Apache no es Viral, permite enlazar con codigo de
 distinta licencia y distribuir cambios con distinta licencia. (manteniendo los cr\'editos) ,
  es incompatible con la GPL.

\section{Mozilla y Thunderbird y Firefox OS}

Antes de 1998 Netscape era uno de los grandes navegadores del momento con casi el 80\% de cuota de mercado, en sus inicios la web
era vista como algo acad\'emico, pero poco a poco se empez\'o a ver su potencial, Microsoft introdujo Internet Explorer en el 
sistema operativo Windows y acab\'o teniendo un 90\% de cuota de mercado.
\\\\
En 1998 la cuota de mercado de Netscape era muy baja y por ello deciden hacer un movimiento arriesgado, liberar el
c\'odigo fuente. Se crea mozilla.org (Mozilla era el nombre interno de Netscape) una organizaci\'on que deb\'ia mantenerlo.
 \\\\
Hubo reacciones de escepticismo y de entusiasmo al mismo tiempo ,hab\'ia gente que no lo entend\'ia, hubo gente que se entusiasm\'o. 
Hubo que desechar el c\'odigo entero porque hab\'ia mezcla de licencias, se decide en muchas partes reescribir el c\'odigo desde cero.
Dos años despu\'es de la liberaci\'on apareci\'o Netscape 6, la primera versi\'on del navegador desde la liberaci\'on del c\'odigo fuente.
\\\\
En 2002 se liber\'o la versi\'on Mozilla 1.0 , en el 2003 se cre\'o la fundaci\'on Mozilla formada por los colaboradores
antiguos de Netscape , las mismas personas que hab\'ian estado desarrollando el navegador los \'ultimos cinco años.
\\\\
Debido a que el proyecto empez\'o a conseguir bastante dinero tuvieron que crear la Mozilla Corporation en el año 2005, que 
depend\'ia directamente de la fundaci\'on.
\\\\
En 2008 surge una nueva divisi\'on, Mozilla Messaging , con pocos empleados , estaban focalizados en el desarrollo de Thunderbird, 
ahora esos empleados est\'an integrados en Moco.
\\\\
El objetivo de Mozilla es que los usuarios tengan la opci\'on de elegir como quieren navegar por Internet y que la 
navegaci\'on por la web sea un estandar , actualmente Mozilla tiene unos 800 empleados, la mayor\'ia 
son colaboradores. 
\subsection{ Firefox OS} % in english 
Firefox OS is a new mobile operating system developed by Mozilla's Boot to Gecko (B2G) project. It uses a Linux kernel and boots into a Gecko-based runtime engine, which lets users run applications developed entirely using HTML, JavaScript,and other open web application APIs.
\\\
In 2011, Dr. Andreas Gal, Director of Research at Mozilla Corporation, announced the "Boot to Gecko"  (B2G) to pursue the goal of building a complete, standalone operating system for the open web “  \footnote{\url{https://wiki.mozilla.org/B2G}}
\\\
On July 2012, Boot to Gecko was rebranded as 'Firefox OS',after Mozilla's well-known desktop browser, Firefox.
\\\\
The first release  on February 2013 .Mozilla announced at Barcelona, Spain at  Mobile World Congress that the first wave of Firefox OS devices will be available to consumers in Brazil, Colombia, Hungary, Mexico, Montenegro, Poland, Serbia, Spain and Venezuela. Firefox have also announced that LG Electronics, ZTE, Huawei and TCL Corporation have committed to making Firefox OS devices. 

\section{GNU}

GNU fue creada por Richard Stallman en 1983 con el objetivo de escribir un sistema operativo tipo unix que fuera software libre , la FSF fue 
creada un año despu\'es para dar soporte legal a GNU. Porqu\'e tipo Unix?, todo por razones t\'ecnicas,
 Unix era popular pero no tanto como VMS,  Richard escogi\'o unix por estar compuesto de utilidades
  independientes y por ser algo m\'as f\'acil de imitar o clonar que VMS, que era mucho m\'as compacto.
\\    
Lanz\'o el GNU Manifesto e invit\'o a cualquier persona a unirse con el, en sus inicios se di\'o cuenta de un pequeño problema: el desarrollo
 de GNU es global, no se circunscribe a un \'area geogr\'afica, hoy en d\'ia no se puede hacer una entidad legal que
  valga para todo el planeta por eso GNU no tiene entidad legal y nunca la ha tenido pero por temas de licencias
   tuvo que crear la FSF dentro de EEUU para poder recaudar donaciones y poder pagar abogados para litigios 
   legales etc.\\
\\
La primera utilidad de GNU fue el Emacs y sigui\'o con el compilador gcc creado a partir de Pastel, un compilador 
libre de Pascal ese fue el gcc1, fue creando sucesivas versiones de gcc y empez\'o a usar software abierto de terceros como latex, xwindow etc.\\
\\
En 1990 ya ten\'ian un mont\'on de utilidades pero faltaba el n\'ucleo, Hurd (el n\'ucleo desarrollado por GNU) es un n\'ucleo basado en un
 microkernel y era demasiado ambicioso, linux apareci\'o en 1991 es un kernel monol\'itico y funcionaba porque 
 era menos ambicioso que Hurd y se decidi\'o integrar linux dentro de GNU, glic (la libreria de GNU) se adapt\'o al nuevo kernel 
 y la FSF contrat\'o a varias personas para realizar todo el trabajo, resultado: GNU/Linux.
\\\\
Se empez\'o a pensar en el desarrollo de una interface gr\'afica , hicieron una
     implementaci\'on conocida como COCOA que llamaron GNUstep , y este fue el escritorio oficial
      de GNU, a finales de los 90 hubo gente en GNU que pensaba que la interfaz tipo Next no era
       apetecible a los usuarios porque a la gente le gustaba windows95,  y surgieron proyectos como GNOME y KDE
       que eran estilo windows para dotar a Linux de un entorno gr\'afico.
 \\\\
En los 90 tambi\'en surge el movimiento open source, 
hubo forks serios: egcs y Xemacs, el primero era un fork de gcc, aparecieron como cr\'itica al modelo
 catedral y por eso surgieron grupos de desarrolladores de gcc y emacs que se fueron, emacs 19
  tardaba tanto en salir que varias personas decidieron hacer su versi\'on a parte llamada Xemacs , la versi\'on 2.95 de gcc 
  tardaba tambi\'en tanto en salir que pas\'o lo mismo, en principio para hacer un fork de manera temporal llamado egcs fusion\'andose
  ambas versiones al final pero con XEmacs y Emacs no ocurri\'o lo mismo.
\\\\
2000: se usan lenguajes de alto nivel, perl python, todo se volvi\'o m\'as lento y pesado , surgieron
 muchas aplicaciones de red, GNUutils, inetutils, hubo un gran inter\'es en el mundo de la seguridad
  GNUtls, GNUpg, y la aparici\'on de GPLv3, LGPLv3 AGPLv3,  hubo que actualizar las licencias de m\'as
   de 400 programas con la peculiaridad de que la GPLv3 era incomplatible con la GPLv2 por definici\'on .
    La GPLV3 se hizo porque con la v2 la gente podr\'ia hacer trampas  y adem\'as se hizo uso de la
     multilicencia con el GPLv3  o superior.Aparece tambi\'en la licencia AGPLv3 que generaliza el concepto de distribuci\'on
      para que se acople a los programas web.
\\\\
En los 2000 tambien hubo cambios organizativos en GNU, a finales aparece un grupo de mantenedores
 llamdado Rabbit Herd que empez\'o a cuestionarse cosas y discutir, hicieron un documento y fue
  presentado a Richard Stallman y a ra\'iz de eso surgieron dos revoluciones una la GNU Advisory
   committe y la otra la GNU Hackers Meetings
\\\\
En  2010 y posteriores hay mas cambios organizativos y aparecier\'on nuevas versiones de paquetes como GNU mediagoblin (youtube de GNU),
 GNU social y una tendencia a las redes y a la seguridad de las mismas.

\section{The Document Foundation}

The Document Foundation es una organizaci\'on creada por antiguos miembros del proyecto OpenOffice , su creaci\'on
fue anunciada en el 2010 y surgi\'o como respuesta a la adquisici\'on de Oracle del proyecto Openoffice y la
obligaci\'on impuesta por Oracle de tener que cederle los derechos de autor de todo lo que se desarrollara.
\\\\
Cuando se cre\'o la fundaci\'on por parte de 33 miembros del proyecto OpenOffice, Oracle invit\'o a los
mismos a abandonar el proyecto por conflicto de intereses.
\\\\
En el 2011, Oracle decidi\'o deshacerse del proyecto, y lo cedi\'o a la Apache Software Foundation, d\'andose
la situaci\'on de tener actualmente dos suites de oficina muy similares y las dos dentro del mundo
del software libre .

\section{Liferay}

Liferay es un portal de gesti\'on de contenidos desarrollado por Brian Chan en Java, lo inici\'o en el año 2000 al darse cuenta de que 
las actuales soluciones en cuanto a portales de contenidos eran excesivamente caras y en sus horas libres se puso a desarrollar el
suyo propio, a los dos años enseñ\'o en la empresa en la que trabajaba lo que hab\'ia estado realizando pero no qued\'o satisfecho
con el dinero que le podr\'ian ofrecer por el, as\'i que decidi\'o liberarlo, coloc\'o todo el proyecto en la forja Sourceforge bajo 
licencia MIT y dado que en el mundo java los gestores de contenido eran muy pobres surgieron empresas que empezaron a utilizarlo y poco
a poco empezaron a ponerse en contacto con Brian porque necesitaban soporte , para poder dar este servicio Brian se puso en contacto
con amigos suyos para que le ayudaran y as\'i casi sin darse cuenta desarroll\'o un modelo de negocio de consultor\'ia alrededor de
un proyecto de software libre.
\\\\
En 2004 se fund\'o la empresa Liferay como tal, tuvo un gran crecimiento, multitud de empresas empezaron a utilizar Liferay y
su equipo de colaboradores creci\'o hasta las cuarenta personas, el problema era que el porcentaje de recursos asignados al
producto a\'un era pequeño, se deb\'ia de desarrollar m\'as. Recibieron ofertas de empresas capital riesgo pero las rechazaron buscando
un modelo de crecimiento m\'as estable basado en los beneficios que fuera creando la empresa.
\\\\
La empresa ha ido teniendo un crecimiento constante, en el 2007 la empresa abri\'o oficinas en Asia, China, España e India y han ido
surgiendo gran n\'umero de versiones la u\'ltima en aparecer ha sido la versi\'on 6.1 liberada en Enero de 2012.

\section{KDE}

Matthias Ettrich es el creador en 1997 del proyecto KDE con el objetivo de crear una interface gr\'afica 
unificada para los sistemas Unix, la primera versi\'on de KDE , la 1.0 apareci\'o en 1998 y contaba con
una gran n\'umero de utilidades. El proyecto KDE ha estado ligado desde un principio a la herramienta
gr\'afica Qt que se licenciaba de manera dual con licencia privativa lo que supuso alg\'un que otro
quebradero de cabeza para los desarrolladores de KDE hasta que en el 2000 Trolltech (la creadora de Qt) 
distribuy\'o las librer\'ias que eran utilizadas para hacer el escritorio con licencia GPL.
\\\\
En el año 2000 se lanz\'o la versi\'on 2.0 , KDE fue reescrito por completo con grandes mejoras tecnol\'ogicas
como el motor de renderizado de HTML KHTML, el protocolo de comunicaci\'on de escritorio DCOP y el gestor de archivos
y navegador Konqueror.
\\\\
En el 2002 apareci\'o la versi\'on 3.0 basado en Qt 3 , los cambios en la API entre la versi\'on 2 y la 3
no eran muy grandes, hubo unas cuantas mejoras visuales que mejoraban el aspecto general del escritorio.
\\\\
La versi\'on 4.0 se lanz\'o en el 2008 y fue una versi\'on en la que se reescribi\'o el c\'odigo por completo,
se basaba en la versi\'on 4.3 de Qt y su caracter\'istica principal es que se modific\'o considerablemente el
aspecto del escritorio , se incluyeron nuevos frameworks como Phonon (interfaz multimedia de KDE), Solid, una
API para redes o Decibel un framework de comunicaci\'on.

\section{GNOME}

En 1997 Matthias Ettrich fund\'o el proyecto KDE, en 1997 Miguel de Icaza anunci\'o el 
proyecto Gnome , en aquellos momentos el sistema operativo de escritorio estrella
era Windows 95, todo el mundo se hab\'ia acostumbrado a el, ambos proyectos, Gnome
y Kde prentend\'ian realizar un escritorio alternativo y surgieron ambos debido a que
KDE que apareci\'o primero estaba basado en una herramienta gr\'afica llamada Qt que no 
era libre y debido a eso Miguel de Icaza en 1997 se propuso realizar un desktop totalmente
libre e inici\'o el proyecto GNOME .
\\\\
Gnome significa Gnu Objetc Model Environment, para desarrollarlo era necesario crear
un sistema de componentes que permitiera a las aplicaciones hablar entre ellas insipir\'andose
en el modelo de componentes COM de Windows y utilizando una implementaci\'on de Corba.
\\\\
Miguel junto con su compañero de universidad de M\'exico Mena empiezan a diseñar Gnome,
KDE ya llevaba un año en marcha y empezaron a buscar algo como Qt de lo que partir para no
empezar desde cero, lo primero que hicieron fue contactar con el fabricante de Qt Troll
para ver si estar\'ian dispuestos a liberar Qt, pero estos se negaron porque viv\'ian
de la versi\'on de Qt privativa que distribu\'ian .
\\\\
No encontraban un toolkit que pudieran usar hasta que Mena pens\'o en GTK (Gimp Toolkit), un toolkit que 
estaba utilizando para una aplicaci\'on de diseño gr\'afico totalmente libre llamada Gimp
y desarrollaron Gnome sobre esa plataforma.
\\\\
Los primeros pasos de Gnome fueron dif\'iciles, la primera versi\'on la 0.10 era un
\'unico paquete con librer\'ias y el panel de Gnome.
\\\\
La 0.20 fu\'e realmente la primera versi\'on estable y distribuible, estaba formada por
varios paquetes : core, admin, games, libs y media.
\\\\
La 0.99 surgi\'o en Noviembre de 1998
\\\\
La 1.0 en Marzo de 1999, y fu\'e un desastre ten\'ia muchos bugs y era inestable, por
esas fechas empezaron a surgir empresas interesadas en el proyecto y con intenci\'on de 
financiarlo.
\\\\
En Marzo de 2000 se hace el primer meeting de desarrolladores : GUADEC en Paris y
es financiado por las principales distribuiciones de Linux como RedHat,Suse etc.
\\\\
La versi\'on 1.2 surgi\'o en ese año como la m\'as usable hasta la fecha y en Agosto se crea
la Gnome Foundation con el apoyo de varias organizaciones como la FSF.
\\\\
En el 2001 sacaron la versi\'on 1.4 con Bonobo, Nautilus, Evolution etc y se celebr\'o
la segunda GUADEC en Copenague focalizado en la versi\'on Gnome 2.0 que intentaba solucionar
los problemas de la 1.4 como la falta de intregraci\'on , demasiadas opciones de configuraci\'on
, no muy usable por usuarios finales etc. Gnome 2.0 introduce una nueva plataforma de
desarrollo, introdujeron un framework de accesibilidad desarrollado por Sun Microsystems
, apareci\'o Pango que es la librer\'ia de internacionalizaci\'on y de fuentes,
aparece tambi\'en Gnome-Vfs una librer\'ia de acceso a sistemas de ficheros virtuales. 
Con Gnome 2 tambi\'en crearon su propio sistema de objetos: GObject y crearon un
Release Team para gestionar las sucesivas versiones de Gnome.
\\\\
Gnome 3 surge con cambios fundamentales, se renueva totalmente el Gnome Shell, 
se crea una nueva plataforma de desarrollo GTK3 que ha evolucionado mucho respecto
a GTK2 , plataforma que sin embargo siguen manteniendo para dar soporte a los usuarios
que continuen utiliz\'andola. 

\section{Canonical}

Canonical es una empresa que fue fundada por Mark Shuttleworth en el 2004 , Mark naci\'o en Sud\'africa (1973), en Welkom (Free State), pero creci\'o en Ciudad
Del Cabo, su amor por la tecnolog\'ia y la inovaci\'on le llevo a estudiar finanzas e Inform\'atica en
la Universidad de Ciudad Del Cabo (UCT), desde muy joven fue un gran emprendedor, el mismo
año que terminaba la universidad fund\'o la empresa Thawte que fue todo un \'exito y con ella y su
venta se hizo con unos increibles beneficios que le han permitido llevar a cabo una gran cantidad de
proyectos todos ellos relacionados con el mundo del software libre, adem\'as durante la d\'ecada de los
90 fue desarrollador de Debian (colabor\'o tambi\'en con el proyecto Apache)
\\\\
En 1995 fund\'o la compañ\'ia Thawte como empresa de consultor\'ia en Internet que se centr\'o en
la seguridad en la red y en el comercio electr\'onico especializ\'andose en certificados digitales y criptograf
\'ia (implantando la infraestructura de clave p\'ublica en el mundo de las transacciones cifradas
y autenticadas por Internet), fue una de las primeras empresas de seguridad en ser reconocidas por
Microsoft y Netscape para la certificaci\'on de sitios web , en 1999 VeriSign compr\'o Thawte por 575
millones de d\'olares,
\\\\
A principios de 2004, Mark desde su empresa , Cannonical Ltd fund\'o el proyecto Ubuntu (creando
tambi\'en la Fundaci\'on Ubuntu), d\'andose a conocer el 8 de Julio de 2004 como una distribuci\'on
Linux basada en Debian , el proyecto arranc\'o con una financiaci\'on incial de 10 millones de d\'olares
y un grupo de desarrolladores de Debian que perdieron la ilusi\'on en ese proyecto (criticado por
su esquema organizativo con excesiva burocracia donde la toma de decisiones era extremadamente
compleja) y decidieron embarcarse en Ubuntu con el objetivo de crear un sistema operativo
de escritorio. basado en Linux (tomando a Debian como distribuci\'on madre) y distribuirlo como
software libre con la m\'axima de hacer un entorno f\'acil de utilizar intentado llegar a la mayor
cantidad de usuarios posible , Ubuntu fue pensada como una iniciartiva que en un futuro pudiera
autofinanciarse.
Su primera versi\'on es del 20 de octubre de 2004 y se calcula que actualmente tiene una presencia
de un 49\% dentro de las distribuciones Linux, Ubuntu es una palabra africana muy antigua
que significa ``humanity to others'', tambi\'en significa ``Yo soy lo que soy debido a lo que todos
somos''
\\\\
Los paquetes de Ubuntu est\'an basados en la rama inestable de Debian Muchos desarrolladores
de Ubuntu tambi\'en mantienen paquetes clave en Debian. Ubuntu coopera con Debian devolviendo
cambios y mejoras en el c\'odigo, aunque existen cr\'iticas sobre las escasas aportaciones y problemas
de compatibilidad que hay entre ambas distribuciones.

\end{document}
